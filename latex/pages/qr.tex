\chapter{QR 分解}
用 Matlab 自带的函数 \lstinline|rand()| 创建一个大小为 $5\times 5$ 的随机矩阵


\begin{lstlisting}[language=Matlab]  
>> A=rand(5,5)*20
A =
     1.6694   16.0673   13.1372   19.6813    9.7938
     2.6634    1.2094   12.5595    3.3434    6.7899
     3.4678    7.9852    5.8397    2.1243   19.0326
     7.8188   10.5375    8.6330    7.4482   18.4066
    16.6276    8.3360    0.3097    3.9624    1.0535
\end{lstlisting}


向下取整后作为项目测试用的矩阵$A$
\begin{equation}
    A=\begin{bmatrix}
        1 & 16  & 13  & 19  & 9  \\
        2  & 1 & 12 & 3  & 6  \\
        3  & 7  & 5 & 2 & 19 \\
        7  & 10  & 8  & 7 & 18 \\
        16  & 8 & 0 & 3 & 1
    \end{bmatrix}
\end{equation}

随机产生列向量 $b$
\begin{lstlisting}[language=Matlab]  
>> b=rand(5,1)*20
b =
    14.7572
    5.3824
    8.4567
    10.9574
    18.8547
\end{lstlisting}

取 $b=[14,5,8,10,18]$

\section{QR 分解的 Matlab 实现及结果验证}
\subsection{矩阵的 QR 分解}



设置矩阵$A$
\begin{lstlisting}[language=Matlab]  
>> A = [14, 8, 0, 4, 4; 4, 10, 18, 9, 9;
        2, 1, 14, 19, 12; 5, 5, 9, 10, 13; 6, 16, 11, 10, 7]
A =
    14     8     0     4     4
     4    10    18     9     9
     2     1    14    19    12
     5     5     9    10    13
     6    16    11    10     7
\end{lstlisting}

调用 Matlab 自带的函数 \lstinline|qr()| 进行矩阵分解,\lstinline|[Q,R] = qr(A)| 函数可以对 $m\times n$ 矩阵 $A$ 执行 $QR$ 分解,满足 $A = QR$。因子 $R$ 是 $m×n$ 上三角矩阵,因子 $Q$ 是 $m\times m$ 正交矩阵。
\begin{equation}
    A = QR
\end{equation}

\begin{lstlisting}[language=Matlab]  
>> [Q,R] = qr(A)
\end{lstlisting}

为了验证结果,我们将上三角矩阵 $U$ 和下三角矩阵 $L$ 相乘
\begin{lstlisting}[language=Matlab]  
>> 
\end{lstlisting}
得到的结果矩阵 $B$ 与待分解矩阵一致


此外,该函数 \lstinline|[Q,R,P] = qr(A)| 还可以返回一个置换矩阵 $P$,并满足 $AP = QR$。在此语法中,$L$ 是单位下三角矩阵,$U$ 是上三角矩阵。
\begin{lstlisting}[language=Matlab]  
>> [Q,R,P] = qr(A)
\end{lstlisting}

\begin{lstlisting}[language=Matlab]  
>> A*P = Q*R
\end{lstlisting}


\subsection{用 LU 分解对线性方程组求解}
假定需要求解的方程组为 $Ax=b$,即
\begin{equation}
    \begin{bmatrix}
        14 & 8  & 0  & 4  & 4  \\
        4  & 10 & 18 & 9  & 9  \\
        2  & 1  & 14 & 19 & 12 \\
        5  & 5  & 9  & 10 & 13 \\
        6  & 16 & 11 & 10 & 7
    \end{bmatrix}
    x = 
    \begin{bmatrix}
        12 \\ 16 \\ 9 \\ 8 \\ 16
    \end{bmatrix}
\end{equation}

求解上述方程组的过程如下
\begin{lstlisting}
>> b=[12;16;9;8;16];
>> [L,U,P] = lu(A);
>> y = L\(P*b)
\end{lstlisting}

验证结果
\begin{lstlisting}
>> D=A*x
\end{lstlisting}


\section{LU 分解的 C++ 实现及结果验证}
\subsection{矩阵的 LU 分解}
\subsection{用 LU 分解对线性方程组求解}